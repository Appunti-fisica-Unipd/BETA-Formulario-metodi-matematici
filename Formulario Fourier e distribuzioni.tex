\documentclass[a4paper, 11pt]{article}
\usepackage[utf8]{inputenc}
\usepackage[italian]{babel}
\usepackage{fullpage}
\usepackage{graphicx}
\usepackage{float}
\usepackage[justification=centering]{caption}
\usepackage{multirow}
\usepackage[colorlinks,allcolors={blue}]{hyperref}
\usepackage{amsmath}
\usepackage{verbatim}
\usepackage{mdframed}
\usepackage{xcolor}
\setlength{\parindent}{0pt}
\usepackage{chngcntr}
\usepackage{braket}

\counterwithin{equation}{section}

\newcommand{\F}[0]{\overset{F}{\longrightarrow}}{}
\newcommand{\Ff}[0]{\mathcal{F}}{}
\newcommand{\dx}[0]{\frac{d}{dx}}{}

\begin{document}

\title{Formulario di metodi matematici per le distribuzioni e le trasformate di Fourier}
\author{Pietro Scapolo}
\date{Anno 2021-2022}
\maketitle

\hrulefill
\begin{centering}
\section{Proprietà della trasformata}
\end{centering}
\bigskip

Linearità:
\begin{equation}
    a \psi(x) + b \chi(x) \F a \Ff[\psi](k) + b \Ff[\chi](k)
\end{equation}
\bigskip

Traslazione:
\begin{equation}
    \psi (x- a) \F e^{-ika} \Ff [\psi](k)    
\end{equation}
\bigskip

Anti traslazione:
\begin{equation}
    e^{iax} \psi(x) \F  \Ff[\psi](k-a)
\end{equation}
\bigskip

Riscalamento:
\begin{equation}
    \psi (ax) \F \frac{1}{|a|} \Ff [\psi]\left(\frac{k}{a}\right)
\end{equation}
\bigskip

Derivata:
\begin{equation}
    \left( \frac{d}{dx} \right)^n \psi (x) \F (ik)^n \Ff [\psi] (k)
\end{equation}
\bigskip

Vicerversa:
\begin{equation}
    x^n \psi(x) \F \left(i \frac{d}{dk} \right)^n \Ff [\psi](k)
\end{equation}
\bigskip

Coniugazione:
\begin{equation}
    \overline{\psi}(x) \F \overline{\Ff[\psi]}(-k)
\end{equation}
\bigskip

Trasformata di funzione pari:
\begin{equation}
   \Ff[\psi](-k) = \Ff[\psi](k)
\end{equation}

\newpage
\hrulefill
\begin{center}
\section{Trasformazioni di Fourier}
\end{center}


\begin{equation}
    1 \F \sqrt{2\pi}\cdot \delta(k)
\end{equation}
\begin{equation}
    e^{iax}= 1 \cdot e^{iax} \F \sqrt{2 \pi} \delta(k-a)
\end{equation}
\begin{equation}
    \delta (x-a) \F \frac{1}{\sqrt{2\pi}} e^{-ika}
\end{equation}
\begin{equation}
    e^{-ax} \F \sqrt{2\pi}\delta(k+ia) 
\end{equation}
\begin{equation}
    e^{-\frac{a}{2}x^2} \F \frac{e^{-\frac{k^2}{2a}}}{\sqrt{a}}
\end{equation}
\begin{equation}
    \cos(ax) \F \sqrt{\frac{\pi}{2}} \bigg( \delta (k-a) + \delta (k+a) \bigg)
\end{equation}
\begin{equation}
    \sin(ax) \F -i \sqrt{\frac{\pi}{2}} \bigg( \delta (k-a) - \delta (k+a) \bigg)
\end{equation}


\bigskip
Con dimostrazione:
\begin{equation}
    P\left(\frac{1}{x}\right)\F -i\sqrt{\frac{\pi}{2}}\cdot sgn(k)
\end{equation}

\begin{equation}
    sgn(x) \F -i \sqrt{\frac{2}{\pi}} P \left( \frac{1}{k} \right) 
\end{equation}

\begin{equation}
    \theta (x) \F \sqrt{\frac{\pi}{2}} \delta(k) - \frac{i}{\sqrt{2 \pi}} P \left( \frac{1}{k} \right) = - \frac{i}{\sqrt{2 \pi}} \frac{1}{k-i \epsilon} 
\end{equation}


\newpage

\hrulefill
\begin{center}
\section{Proprietà delle distribuzioni}
\end{center}
\begin{equation}
    \braket{\overline{\phi}|\psi} = \overline{\braket{\phi|\overline \psi}}
\end{equation}
\begin{equation}
    \braket{\phi (ax) | \psi} = \frac{1}{|a|} \braket{\phi| \psi \left( \frac{x}{a} \right)}
\end{equation}
\begin{equation}
    \braket{\phi(x-a)|\psi}= \braket{\phi | \psi(x+a)}
\end{equation}
\begin{equation}
    \braket{f \phi | \psi} = \braket{\phi | \overline f \psi}
\end{equation}

\vspace{3cm}
\hrulefill
\begin{centering}
\section{Distribuzioni: identità}
\end{centering}

\begin{equation}
    xf(x)=xg(x) \Rightarrow f(x) = g(x) + c \delta (x)  
\end{equation}

\begin{equation}
    \Pi \left( \frac{x}{2} \right) = \theta (x+1) - \theta (x-1)
\end{equation}
\begin{equation}
    \Pi \left( \frac{x}{2} \right) = \frac{1}{2}  \big( sgn(x+1) - sgn(x-1) \big)
\end{equation}
\begin{equation}
    \braket{P \left(\frac{1}{x} \right) | \psi} = \int_0^{+ \infty} \frac{\psi(x)-\psi(-x)}{x}dx
\end{equation}
\begin{equation}
    x P \left( \frac{1}{x} \right) = 1
\end{equation}
\begin{equation}
    sgn(x)= 2 \theta (x) -1
\end{equation}

\newpage
\begin{centering}
\subsection{Identità di $\delta$}
\end{centering}

\begin{equation}
    x \delta (x) =0
\end{equation}
\begin{equation}
    \braket{\delta^{(n)} (x-a) | \psi} = (-1)^n \psi ^{(n)} (a)
\end{equation}
\begin{equation}
    \overline{\delta (x)} = \delta(x)
\end{equation}
\begin{equation}
    \delta(-x) = \delta(x)
\end{equation}
\begin{equation}
    \delta (ax) = a^{-1} \delta(x)
\end{equation}
\begin{equation}
    \delta(x^2 -a^2) = \frac{1}{2} a^{-1} \bigg( \delta(x-a) + \delta (x+a) \bigg)
\end{equation}
\begin{equation}
    f(x) \delta (x-a) = f(a) \delta (x-a)
\end{equation}

\begin{equation}
    \delta [g(x)] = \sum_{i=1}^N \frac{\delta (x-a_i)}{|g'(a_i)|}
\end{equation}
\emph{Con $a_i$ zeri semplici per $g(x)$.}


\vspace{3cm}
\hrulefill
\begin{centering}
\section{Distribuzioni: derivate notevoli}
\end{centering}

Regola di Leibniz:
\begin{equation}
    \dx (f \phi) = f' \phi + f \phi' 
\end{equation}

\begin{equation}
    \dx log |x| = P \left( \frac{1}{x} \right)
\end{equation}

\begin{equation}
    - \frac{1}{n} \dx P \left( \frac{1}{x} \right) = P \left( \frac{1}{x^{n+1}} \right)
\end{equation}

\begin{equation}
    \dx \theta (x) = \delta (x)
\end{equation}

\begin{equation}
    \dx sgnx (x) = 2 \delta(x)
\end{equation}


\end{document}